\usepackage{amsmath,amssymb}
\usepackage{bm}
\usepackage[dvipdfmx]{graphicx}
\usepackage{verbatim}
\usepackage{wrapfig}
\usepackage{ascmac}
\usepackage{makeidx}
\usepackage{listings}
\usepackage{caption}
\usepackage{multicol}
\usepackage{longtable}
\usepackage{here}
\usepackage{docmute}

\newcommand{\xDiff}[3]{\frac{#1 #2}{#1 #3}}
\newcommand{\xDiffn}[4]{\frac{#1^#4 #2}{#1 #3^#4}}
\newcommand{\D}[1]{\mathrm{d}#1}
\newcommand{\Diff}[2]{\xDiff{\D}{#1}{#2}}
\newcommand{\Diffn}[3]{\xDiffn{\D}{#1}{#2}{#3}}
\newcommand{\Par}[2]{\xDiff{\partial}{#1}{#2}}
\newcommand{\Parn}[3]{\xDiffn{\partial}{#1}{#2}{#3}}
\newcommand{\FDiff}[2]{\xDiff{\delta}{#1}{#2}}
\newcommand{\Int}[3]{\int_{#1}^{#2} #3}
\newcommand{\Sum}[2]{\displaystyle \sum_{#1}^{#2} }
\newcommand{\Div}[1]{\nabla \cdot #1}

\newcommand{\Fig}[4]
{
    \begin{figure}[H]
        \begin{center}
            \includegraphics[#4]{#1}
        \end{center}
        \caption{#2}
        #3 %label
    \end{figure}
}

\newcommand{\FigLess}[3]
{
    \begin{figure}[H]
        \begin{center}
            \includegraphics{#1}
        \end{center}
        \caption{#2}
        #3 %label
    \end{figure}
}

\newcommand{\FigILess}[3]
{
\begin{figure}[H]
    \begin{center}
        \input{#1}
    \end{center}
    \caption{#2}
    #3 %label
\end{figure}
}

\newcommand{\FigIdouble}[2]
{
\begin{figure}[H]
    \begin{minipage}{0.5\hsize}
        #1
    \end{minipage}
    \begin{minipage}{0.5\hsize}
        #2
    \end{minipage}
\end{figure}
}

\newcommand{\FigItriple}[3]
{
\begin{figure}[H]
    \begin{minipage}{0.3\hsize}
        #1
    \end{minipage}
    \begin{minipage}{0.3\hsize}
        #2
    \end{minipage}
    \begin{minipage}{0.3\hsize}
        #3
    \end{minipage}
\end{figure}
}

\newcommand{\FigItripleCap}[5]
{
\begin{figure}[H]
    \begin{minipage}{0.3\hsize}
        #1
    \end{minipage}
    \begin{minipage}{0.3\hsize}
        #2
    \end{minipage}
    \begin{minipage}{0.3\hsize}
        #3
    \end{minipage}
    \caption{#4}
    #5 %label
\end{figure}
}

\newcommand{\FigIdoublebeamer}[2]
{
\begin{columns}
    \begin{column}{0.48\textwidth}
        #1
    \end{column}
    \begin{column}{0.48\textwidth}
        #2
    \end{column}
\end{columns}
}

\newcommand{\FigI}[5]
{
\begin{figure}[H]
    \begin{center}
        \resizebox*{#4}{#5}{\input{#1}}
    \end{center}
    \caption{#2}
    #3 %label
\end{figure}
}

\newcommand{\FigInoCap}[3]
{
\begin{figure}[H]
    \begin{center}
        \resizebox*{#2}{#3}{\input{#1}}
    \end{center}
\end{figure}
}
\makeatletter
\c@MaxMatrixCols=18

\def\Hline{
    \noalign{\ifnum0=`}\fi\hrule \@height 4.\arrayrulewidth \futurelet
\reserved@a\@xhline}

\makeatother
