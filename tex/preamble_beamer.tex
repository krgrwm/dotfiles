%Beamerの設定
%\usetheme{Boadilla}
\usetheme{boxes}
%Beamerフォント設定
\usepackage{txfonts} % TXフォント
% \usepackage[deluxe]{otf} % platexの場合はこちら
\usepackage{pxjahyper} % PDF目次文字化け回避(platexでは不要)
\renewcommand{\familydefault}{\sfdefault}  % 英文をサンセリフ体に
\usefonttheme{structurebold} % タイトル部を太字
\setbeamerfont{alerted text}{series=\bfseries} % Alertを太字
\setbeamerfont{section in toc}{series=\mdseries} % 目次は太字にしない
\setbeamerfont{frametitle}{size=\Large} % フレームタイトル文字サイズ
\setbeamerfont{title}{size=\LARGE} % タイトル文字サイズ
\setbeamerfont{date}{size=\small}  % 日付文字サイズ

% Babel (日本語の場合のみ・英語の場合は不要)
\uselanguage{japanese}
\languagepath{japanese}
%Beamer色設定
\definecolor{UniBlue}{RGB}{0,150,200} 
\definecolor{AlertOrange}{RGB}{255,76,0}
\definecolor{AlmostBlack}{RGB}{38,38,38}
\setbeamercolor{normal text}{fg=AlmostBlack}  % 本文カラー
\setbeamercolor{structure}{fg=UniBlue} % 見出しカラー
\setbeamercolor{block title}{fg=UniBlue!50!black} % ブロック部分タイトルカラー
\setbeamercolor{alerted text}{fg=AlertOrange} % \alert 文字カラー
\mode<beamer>{
    \definecolor{BackGroundGray}{RGB}{254,254,254}
    \setbeamercolor{background canvas}{bg=BackGroundGray} % スライドモードのみ背景をわずかにグレーにする
}

%フラットデザイン化
\setbeamertemplate{blocks}[rounded] % Blockの影を消す
\useinnertheme{circles} % 箇条書きをシンプルに
\setbeamertemplate{navigation symbols}{} % ナビゲーションシンボルを消す
\setbeamertemplate{footline}[frame number] % フッターはスライド番号のみ

%タイトルページ
\setbeamertemplate{title page}{%
    \vspace{2.5em}
    {\usebeamerfont{title} \usebeamercolor[fg]{title} \inserttitle \par}
    {\usebeamerfont{subtitle}\usebeamercolor[fg]{subtitle}\insertsubtitle \par}
    \vspace{1.5em}
    \begin{flushright}
        \usebeamerfont{author}\insertauthor\par
        \usebeamerfont{institute}\insertinstitute \par
        \vspace{3em}
        \usebeamerfont{date}\insertdate\par
        \usebeamercolor[fg]{titlegraphic}\inserttitlegraphic
    \end{flushright}
}


% Tikz
\usepackage{tikz}
\usetikzlibrary{positioning,shapes,arrows}


\AtBeginSection[]{
    \frame{\tableofcontents[currentsection, hideallsubsections]} %目次スライド
}

% caption 日本語
\renewcommand{\figurename}{図}
\renewcommand{\tablename}{表}

% zu, hyou wo kesu
\usepackage[labelformat=empty,labelsep=none]{caption}

% caption に番号追加
\setbeamertemplate{caption}[numbered]

\newcommand{\sizedblock}[3]
{
    \begin{minipage}{#1}
        \begin{block}{#2}
            #3
        \end{block}
    \end{minipage}
}

\newcommand{\himath}[2][yellow]{\tikz[baseline=(x.base)]{\node[rectangle,rounded corners,fill=#1, opacity=0.4, text opacity=1](x){$\displaystyle{#2}$};}}
\newcommand{\himathalpha}[3][yellow]{\tikz[baseline=(x.base)]{\node[rectangle,rounded corners,fill=#1, opacity=#2, text opacity=1](x){$\displaystyle{#3}$};}}
\newcommand{\himathcapalpha}[4][yellow]{\tikz[baseline=(x.base)]{\node[rectangle,rounded corners,fill=#1, opacity=#2, text opacity=1](x){$\displaystyle{#4}$} node[below=-0.5ex of x, color=#1, align=left]{#3};}}


\newcommand{\mathcap}[3][yellow]{\tikz[baseline=(x.base)]{\node(x){$\displaystyle{#3}$} node[below=-0.5ex of x, color=#1]{#2};}}

\usepackage[absolute,overlay]{textpos}

\usepackage{txfonts}
\usefonttheme{professionalfonts}%普通(TX)の数式フォント(宣言しないとsansになる!)
